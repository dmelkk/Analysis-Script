\documentclass[../Analysis1_script.tex]{subfiles}
\begin{document}

\subsection{Fakultät}

\begin{definition}{(Fakultät)}
	Die Funktion $n \in \mathbb {N}_0 \mapsto n! \in \mathbb {N}$ ist definiert durch
	\[\begin{aligned}[]
		0! = 1,\ n! = \prod _{k=1}^n k.
	\end{aligned}\]
Die Zahl $n!$ wird als $n$–\textbf{Fakultät} oder $n$–\textbf{Faktorielle} bezeichnet. 
\end{definition}

Insbesondere (nämlich per Definition des Produkts) gilt also die rekursive Formel
\[\begin{aligned}[]
	(n+1)! = (n!)\cdot (n+1)
\end{aligned}\]
für alle $n\in \mathbb {N}_0$. Wir werden dieser Funktion in vielen weiteren Funktionen und Ausdrücken begegnen. Sie hat jedoch auch für sich gesehen eine (kombinatorische) Bedeutung.

\begin{lemma}{(Kordinalität der Menge der Permutation einer endlichen Menge)}
	Für $n\in \mathbb {N}$ ist $n!$ die Kardinalität der Menge $\mathcal {S}_n$ der bijektiven Abbildungen $\sigma :\left \lbrace {1,\ldots ,n} \right \rbrace \to \left \lbrace {1,\ldots ,n} \right \rbrace$ (auch Permutationen von $\left \lbrace {1,\ldots ,n} \right \rbrace$ genannt). 
\end{lemma}

Intuitiv ausgedrückt gibt es also genau $n!$ verschiedene Möglichkeiten die Menge $\left \lbrace {1,\ldots ,n} \right \rbrace$ zu sortieren oder auch $n!$ Möglichkeiten für die verschiedenen Reihenfolgen, wenn alle n nummerierte Bälle zufällig aus einer Urne gezogen werden.

\begin{remark}
	Zu $n\in \mathbb {N}$ bildet die Menge $\mathcal {S}_n$ zusammen mit der Verknüpfung von Elementen $(\sigma ,\tau )\in \mathcal {S}_n^2 \mapsto \sigma \circ \tau \in \mathcal {S}_n$ eine Gruppe (die \textbf{symmetrische Gruppe}). 
\end{remark}

\begin{proof}
	Wir beweisen die Aussage per Induktion. Für $n=1$ gibt es genau eine (bijektive) Abbildung $\left \lbrace {1} \right \rbrace \to \left \lbrace {1} \right \rbrace$, was den Induktionsanfang darstellt. 
	
	Angenommen die Aussage des Lemmas gilt bereits für ein $n\in \mathbb {N}$. Wir betrachten nun eine Permutation $\sigma$ von $\left \lbrace {1,\ldots ,n+1} \right \rbrace$. Falls $\sigma (n+1) = n+1$ gilt, so erhalten wir mittels Einschränkung auf $\left \lbrace {1,\ldots ,n} \right \rbrace$ eine bijektive Abbildung $\sigma ': k \in \left \lbrace {1,\ldots ,n} \right \rbrace \to \left \lbrace {1,\ldots ,n} \right \rbrace$ in $\mathcal {S}_n$. Umgekehrt können wir für jedes $\sigma '\in \mathcal {S}_n$ eine Fortsetzung $\sigma \in \mathcal {S}_{n+1}$ mit $\sigma (n+1)=n+1$ definieren. Daher wissen wir also per Induktionsannahme, dass es n! Abbildungen $\sigma \in \mathcal {S}_{n+1}$ mit $\sigma (n+1)=n+1$ gibt. Wir bezeichnen die Menge aller solcher Permutation in $\mathcal {S}_{n+1}$ mit 
	\[H=\left \lbrace {\sigma \in \mathcal {S}_{n+1}} \mid {\sigma (n+1)=n+1}\right \rbrace ,\]
	so dass $|H|=n!$
	Die Menge $\mathcal {S}_{n+1}$ lässt sich wie folgt partitionieren:
	 \[\mathcal {S}_{n+1} = \bigsqcup _{k=1}^{n+1}P_k\text { mit } P_k=\left \lbrace {\tau \in \mathcal {S}_{n+1}} \mid {\tau (n+1) = k}\right \rbrace\]
für $k=1,\ldots ,n+1$. Wir behaupten nun, dass die Mengen $P_k$ auf der rechten Seite alle Kardinalität $n!$ haben (für $k=n+1$ ist dies bereits bekannt, da $P_{n+1}=H$). Dies impliziert 
	 \[|\mathcal {S}_{n+1}| = (n!)\cdot (n+1) = (n+1)!\]
und damit nach vollständiger Induktion das Lemma.
\end{proof}

\subsection{Binomialkoeffizienten}

\subsection{Der binomische Lehrsatz}

\subsection{Eine Summe von Binomialkoeffizienten}



\end{document}