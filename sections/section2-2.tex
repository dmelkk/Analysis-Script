\documentclass[../Analysis1_script.tex]{subfiles}
\begin{document}


\begin{definition}{(Polynomfunktionen)}
	Eine \textbf{Polynomfunktion} auf $\mathbb{C}$ ist ein Funktion der Form
	\[f : z \in \mathbb{C} \mapsto \sum_{k=0}^n a_k z^k \in \mathbb{C}\]
	für $n \in \mathbb{N}_0$ und $a_0, \ldots, a_n \in \mathbb{C}$. Die Zahlen $a_0, \ldots, a_n \in \mathbb{C}$ heissen die \textbf{Koeffizienten} von $f$. Das grösste $k \in \{0, \ldots, n\}$ mit $a_k \neq 0$ ist der \textbf{Grad} $\deg(f)$ der Polynomfunktion $f$ und $a_{\deg(f)}$ ist der \textbf{Leitkoeffizient} oder \textbf{führende Koeffizient} von $f$. Falls keine solche $k$ existiert, das heisst, falls $f$ Polynomfunktion $z \in \mathbb{C} \mapsto 0z^0 = 0 \in \mathbb{C}$ ist, so nennt man die Polynomfunktion die \textbf{Null} und setzt den Grad auf $-\infty$. Eine Polynomfunktion der Form $z \in \mathbb{C} \mapsto 0z^0 = 0 \in \mathbb{C}$ für $a_0 \in \mathbb{C}$ wird auch \textbf{konstant} gennant und kurz mit $a_0$ bezeichnet. Eine Polynomfunktion mit Grad $\leq 1$ wird \textbf{affin} oder \textbf{linear} gennant. Eine Polynomfunktion der Form $z \in \mathbb{C} \mapsto a_k z^k \in \mathbb{C}$ für $k \in \mathbb{N}_0$ heisst ein \textbf{Monom}. Wir sagen, dass ein Polynomfunktion \textbf{reell} ist, wenn die Koeffizienten reellt gewählt werden können. Wir werden eine reelle Polynomfunktion auch mit der zugehörigen Funktion von $\mathbb {R}$ nach $\mathbb {R}$ identifizieren. 
\end{definition}

\begin{definition}{(Polynome)}
	Sei $\mathbb{K}$ ein beliebiger Körper. Ein \textbf{Polynome} $f$ über $\mathbb{K}$ ist ein formaler Ausdruck der Form $\sum_{k=0}^n a_k T^k$ für $n \in \mathbb{N}_0$ und Koeffizienten $a_0, \ldots, a_n \in \mathbb{K}$. Hierbei ist $T$ ein Symbol, das man auch als \textbf{Variable} bezeichnet und das verwendet wird, um die Koeffizienten von einander zu trennen. Wir schreiben auch $T^1 = T$ und $aT^0 = a$ für alle $a \in \mathbb{K}$. Weiter darf ein Summand der Form $0T^k$ für $k\in \mathbb {N}_0$ aus der Summe entfernt werden. Wir definieren den \textbf{Polynomring} $\mathbb{K}[T]$ als die Menge der Polynome über $\mathbb {K}$ in der Variablen $T$ mit Addition und Multiplikation gegeben und verwenden ebenso die Begriffe Grad, Koeffizient, etc.
\end{definition}

\begin{example}{(Polynome auf endliche Körpern)}
	Wir betrachten für den Körper $\mathbb {F}_2$ mit zwei Elementen die Polynomfunktionen 
	\[\begin{aligned}[]
		f:a\in \mathbb {F}_2 \mapsto a^3+a+1 \in \mathbb {F}_2,\quad g:a\in \mathbb {F}_2 \mapsto 1 \in \mathbb {F}_2.
	\end{aligned}\]
	Bei $0\in \mathbb {F}_2$ gilt $f(0) = 1 = g(0)$ und an der Stelle $1\in \mathbb {F}_2 gilt f(1) = 1+1+1 = 1$ und $g(1) = 1$. Insbesondere gilt $f=g$, obwohl $f$ und $g$ nicht durch die gleichen Koeffizienten gegeben sind. Wir unterscheiden die Polynome $T^3+T+1\in \mathbb {F}_2[T]$ (mit Grad $3$) und $1\in \mathbb {F}_2[T]$ (mit Grad $0$), obwohl die zugehörigen Polynomfunktionen identisch sind (womit es für diese Polynomfunktion keinen wohldefinierten Grad gibt). 
\end{example}

\begin{proposition}{(Wachstum von Polynomfunktionen und Eindeutigkeit der Koeffizienten)}
	Sei $f(T) \in \mathbb {C}[T]$ ein nicht-konstantes Polynom. Dann gibt es zu jeder positiven reellen Zahl $M > 0$ eine reelle Zahl $R \geq 1$, so dass für alle $z \in \mathbb {C}$ mit $|z| \geq R$ auch $|f(z)| \geq M$ gilt. Insbesondere ist die Zuordnung, die jedem Polynom $f(T) \in \mathbb {C}[T]$ die zugehörige Polynomfunktion $z \in \mathbb {C} \mapsto f(z) \in \mathbb {C}$ zuweist, bijektiv. Dies gilt analog ebenso für reelle Polynome $f(T)\in \mathbb {R}[T]$ und reelle Polynomfunktionen $x\in \mathbb {R}\mapsto f(x)\in \mathbb {R}$
\end{proposition}

\begin{proof}
	Sei $f(T) = a_n T^n + a_{n-1} T^{n-1} + \ldots + a_1 T + a_0 \in \mathbb{C}[T]$ mit $a_0 \neq 0$ und $n \geq 1$. Wir definieren $q(T) \in \mathbb{C}[T]$ durch $q(T) = a_{n-1} T^{n-1} + \ldots + a_1 T + a_0$, womit $f(T) = a^n T_n + q(T)$. Nun behaupten wir, dass die Polynomfunktion $q(z)$ "langsamer wächst als" $a_n z^n$ und schätzen also $|q(z)|$ für $z \in \mathbb{C}$ mit $|z| \geq 1$ nach oben ab:
	\[\begin{aligned}[]
		|q(z)| &= |a_{n-1}z^{n-1}+ \ldots +a_1z+a_0| \\ 
		&\leq |a_{n-1}z^{n-1}|+ \ldots +|a_1z|+|a_0|\\ 
		&= |a_{n-1}||z^{n-1}|+ \ldots +|a_1||z|+|a_0|\\ 
		&\leq (|a_{n-1}|+\ldots + |a_1| + |a_0|) |z|^{n-1} = A |z|^{n-1},
	\end{aligned}\]
	wobei wir $A= |a_{n-1}|+\ldots + |a_1| + |a_0|$ gestzt haben und $|z| \geq 1$ in der Form $|z|^k \leq |z|^{n-1}$ für $k \in \{0, \ldots, n-1\}$ verwendet haben. Mit der umgekehrten Dreiecksgleichung und $f(z) = a_n z^n + q(z)$ gilt somit
	\[\begin{aligned}[]
		|f(z)| \geq |a_nz^n| -|q(z)| &\geq |a_n||z|^n - A |z|^{n-1} \\ 
		&= (|a_n||z|-A) |z|^{n-1} \geq |a_n||z|-A,
	\end{aligned}\]
	falls $|a_n||z| - A \geq 0$ oder äquivalenterweise $|z| \geq \frac{A}{|a_n|}$. Sei nun $M > 0$ beliebig. Dann wählen wir
	\[R = \max \left \lbrace {1,\frac {A}{|a_n|},\frac {A+M}{|a_n|}} \right \rbrace .\]
	Falls nun $z \in \mathbb{C}$ die Ungleichung $|z| \geq R$ erfüllt, dann gilt $|z| \geq 1$ und $|z| \geq \frac{A}{|a_n|}$, wonach obige Unglecihungen ergeben
	\[|f(z)| \geq |a_n||z|-A \geq |a_n|\frac {A+M}{|a_n|} -A = M,\] 
	was die erste Behauptung der Proposition beweist.
	
	Angenommen $f_1(T), f_2(T) \in \mathbb {C}[T]$ sind zwei Polynome, die $f_1(z) = f_2(z)$ für alle $z in \mathbb {C}$ erfüllen. Dann hat das Polynom $g(T) = f_1(T) - f_2(T)$ die Eigenschaft, dass $g(z) = 0$ für alle $z \in \mathbb {C}$ gilt. Falls der Grad des Polynoms $g(T)$ grösser gleich Eins ist, widerspricht dies dem ersten Teil der Proposition. Also ist $g(T)$ konstant, womit $g(T) = 0$ gelten muss und daher sind die Polynome $f_1(T)$ und $f_2(T)$ identisch (d.h. sie haben denselben Grad und dieselben Koeffizienten). Diesen Beweis kann man ebenso für reelle Polynome und die zugehörigen Polynomfunktionen von $\mathbb {R}$ nach $\mathbb {R}$ durchführen.
\end{proof}

\subsection{Polynomdivision}
	Wie wir gesehen haben, existiert auf $\mathbb {N}$ eine Division von $n$ durch $d$ mit Rest gegeben durch $n=qd+r$. Dabei ist der Rest $r$ strikt kleiner (bezüglich $\leq$) als $d$. Division mit Rest gibt es auch für Polynome. Hier hat der Rest bei der Divison von $f$ durch $d$ einen kleineren Grad als $d$. Wir illustrieren dies an einem Beispiel.
	
\begin{example}
\end{example}

\subsection{Nullstellen und Interpolation}

\subsection{Algebraische und tranzendente Zahlen}
\end{document}