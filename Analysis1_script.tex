\documentclass[12pt, twoside, openright]{report}
% ------------- Graphic ------------------------
\usepackage{svg}
\usepackage{graphicx}
\graphicspath{{images/}{../images/}}
% ----------------------------------------------
\usepackage[utf8]{inputenc} % change coding to UTF-8
% ----------- Some math packages ----------------
\usepackage{mathtools}
\usepackage{amssymb}
\usepackage{amsthm} 
% -----------------------------------------------
\usepackage{yfonts} % aditional fonts
\usepackage{enumitem} % smart lists
\usepackage{enumitem, hyperref} % links
\usepackage{subfiles}

% setting depth of counting
% \setcounter{secnumdepth}{3}

% To print the traditional QED (quod erat demonstrandum) at the end of a proof.
\renewcommand\qedsymbol{$\blacksquare$}

% --------- style of additional sections -----------
\theoremstyle{definition} % Theorem style
\newtheorem{theorem}{Theorem}[chapter]

\newtheorem{corollary}{Folgerung}[theorem]

\theoremstyle{definition} % Definition style
\newtheorem{definition}[theorem]{Definition}

\newtheorem{example}{Beispliel}[theorem]

\theoremstyle{definition} % Proposition style
\newtheorem{proposition}[theorem]{Satz}

\theoremstyle{definition} % Axeom style
\newtheorem{axiom}[theorem]{Axiom}

\theoremstyle{definition} % Lemma style
\newtheorem{lemma}[theorem]{Lemma}

\theoremstyle{remark} % Remark style
\newtheorem*{remark}{Remark}
% --------- end of section styling -----------------


\title{
	{Analysis Script}\\
	{\large RWTH Aachen}\\
}

\author{Melkonian Dmytro}
\date{14 October 2018}

\begin {document}
	\maketitle

	\tableofcontents	
	
		
	\chapter{Die reelen Zahlen}
	
		\section{Die Axiome der reellen Zahlen}
		\subfile{sections/section1-1.tex}
		
		\section{Die natürlichen Zahlen}
		\subfile{sections/section1-2.tex}
		
		\section{Die komplexen Zahlen}
		\subfile{sections/section1-3.tex}
		
		\section{Intervalle und der Absolutbetrag}
		\subfile{sections/section1-4.tex}
		
		\section{Maximum und Supremum}
		\subfile{sections/section1-5.tex}
		
		\section{Konsequnezen der Vollständigkeit}
		\subfile{sections/section1-6.tex}
		
		\section{Modelle und Eindeutigkeit der Menge der reellen Zahlen}
	
	\chapter{Funktionen und die reellen Zahlen}
	
		\section{Summen und Produkte}
		\subfile{sections/section2-1.tex}
		
		\section{Polynome}
		\subfile{sections/section2-2.tex}
		
		\section{Der Fakultät und der Binomialsatz}
		\subfile{sections/section2-3.tex}
		
		\section{Reelerwertige Funktionen}
		\subfile{sections/section2-4.tex}
		
		\section{Stetigkeit}
		\subfile{sections/section2-5.tex}
	

\end {document}
