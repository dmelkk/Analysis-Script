\documentclass[../Analysis1_script.tex]{subfiles}
\begin{document}
 
\subsection{Maximum und Minimum}

\begin{definition}{(Maximum)}
	Wir sagen, dass $x_0 = \max(X) \in \mathbb{R}$ das \textbf{Maximum} einer Teilmenge $X \subseteq \mathbb{R}$ ist, falls $\forall x \in X: x \leq x_0$.
	
	Das Maximum einer Teilmenge $X \subseteq \mathbb{R}$ ist eindeutig bestimmt. Denn falls $x_0, x_0'$ beide die Eigenschaften eines Maximums erfüllen, so folgt $x_0 \leq x_0'$ (weil $x_0 \in X$ und $x_0'$ ein Maximum ist) und $x_0' \leq x_0$ (weil $x_0' \in X$ und $x_0$ ein Maximum ist) und damit $x_0 = x_0'$
	
	Ein abgeschlossenes Intervall $[a, b]$ mit Endpunkte $a < b$ in $\mathbb{R}$ hat $b = max([a, b])$ als Maximum. Auch nicht-leere endliche Teilmengen und viele weitere Mengen besitzen ein Maximum. Es gibt jedoch auch Mengen, die kein Maximum bestizt. Beispielweise hat das offene Intervall $(a, b)$ mit Endpunkte $a < b$ in $\mathbb{R}$ kein Maximum.
	
	\textbf{*** Hier muss Beweis sein***}
	
	Des Weiteren kann $\mathbb{R}$ (oder acuh  Intervale der Form $[a, +\infty), (a, +\infty)$ für $a \in \mathbb{R}$) kein Maximum besitzen, da für beliebige $x \in \mathbb{R}$ die Ungleichung $x < x + 1$ gilt und damit $x$ kein Maximum sein kann. 
\end{definition}

\begin{definition}{(Minimum)}
	Wir sagen, dass $x_0 = \min(X)$ das \textbf{Minimum} einer Teilmenge $X \subseteq \mathbb{R}$ ist, falls $\forall x \in X: x \geq x_0$.
	
	Die obige Diskussion lässt sich auf analoge Weise für das Minimum anwenden. Dieses ist also eindeutig bestimmt, muss aber nicht unbedingt existieren.
\end{definition}


\subsection{Supremum und Infimum}

\begin{definition}{(Beschränkheit und Schranken)}
	Eine Teilemenge $X \subseteq \mathbb{R}$ heisst \textbf{von oben beschränkt}, falls es ein $s \in \mathbb{R}$ gibt mit $x \leq s$ für alle $x \in X$. Ein solches $s \in \mathbb{R}$ nennt man in diesem Fall eine \textbf{obere Schranke} von $X$. Die Begriffe "\textbf{von unter beschränkt}" und "\textbf{untere Schranke}" sind analog definiert. Eine Teilmenge $X \subseteq \mathbb{R}$ heisst \textbf{beschränkt}, falls sie von oben und von unter beschränkt ist.
\end{definition}

\begin{proposition}{(Supremum)}
	Sei $X \subseteq \mathbb{R}$ eine von oben beschränkte, nicht-leere Teilmenge. Dann gibt es eine \textbf{kleinste obere Schranke} von $X$, die auch das \textbf{Supremum} $\sup (X)$ gennant wird. Formal gelten also für $s_0 = \sup(X)$ folgende Eigenschaften:
	\begin{enumerate}
		\item{($s_0$ ist eine obere Scharnke)} $\forall x \in X : x \leq s_0$
		\item{($s_0$ ist kleiner gleich jeder oberen Schranke)} $\forall s \in \mathbb{R}: ((\forall x \in X: x \leq s) \implies s_0 \leq s)$
	\end{enumerate} 
	Äquivalenterweise kann $s_0 = \sup(X)$ auch durch (1) und die folgende Bedingung definiert werden:
	\begin{enumerate}[resume]
		\item{(Kleinere Zahlen sind keine oberen Schranken)}$\forall \varepsilon > 0 \,\exists x \in X : x > s_0 - \varepsilon$
	\end{enumerate}
	
	Ein paar wichtige Bemerkungen dazu:
	\begin{itemize}
		\item Falls das Maximum $x_0 = \max(X)$ existiert, dann ist $x_0$ eine obere Schranke von $X$ und ist vielmehr auch die kleinste obere Schranke, also $\max(X) = \sup(X)$. Denn aus $x_0 \in X$ folgt $x_0 \leq s$ für jede obere Schranke $s$ von $X$.
		\item Wenn das Supremum $\sup(X) \in X$, dann ist $\sup(X) = \max(X)$, da das Supremum eine obere Schranke ist. Also ist das Supremum eine  Verallgemeinerung des Maximums einer Menge.
		\item Die Formulierung "kleinste obere Schranke" ist natürlich ein Synonym für das Minimum der oberen Schranken und ist dadurch eindeutig bestimmt, falls es existiert.
	\end{itemize}
\end{proposition}

\begin{proof}
	Nach Annahme ist $X$ nicht-leer und die Menge der oberen Schranken $Y = \{s \in \mathbb{R} | \forall x \in X: x \leq s\}$ ist ebenfalls nict-leer. Das Weiteren gilt für alle $x \in X, s \in Y$, die Ungleichung $x \leq s$. Nach dem Vollständigkeitsaxiom folg daher, dass es $\forall x \in X, s \in Y \, \exists c \in \mathbb{R}: x \leq c \leq s$. Aus der ersten Unglecihung folgt, dass $c$ eine obere Schranke von $X$ ist. Aus der zweiten Unglecihung folgt, dass $c$ kleinste obere Schranke von $X$ ist, und daher erfüllt $c$ sowohl (1) als auch (2).
	
	Wir zeigen nun, dass das Supremum auch durch (1) und (3) charakterisiert wird. Also angenommen $s_0 = \sup(X)$ und $\varepsilon > 0$, dann ist $s_0 - \varepsilon < s_0$. Daher kann $s_0 - \varepsilon$ keine obere Schranke sein und $\exists x \in X: x > s_0 - \varepsilon$. Daher erfüllt $s_0$ auch (3).
	
	Erfüllt $t_0 \in \mathbb {R}$ nun (1) und (3), so ist $t_0$ eine obere Schranke und daher ist $s_0 \leq t_0$ nach Definition von $s_0 = \sup (X)$. Falls $s_0 < t_0$ wäre, dann wäre $s_0 = t_0 - \varepsilon$ für ein $\varepsilon > 0$. Nach der zweiten Eigenschaft von $t_0$ gäbe es ein $x \in X$ mit $x>s_0$, was der Definition von $s_0$ als (kleinste) obere Schranke widerspricht. Deswegen muss $t_0 = s_0$ gelten und $s_0$ ist eindeutig durch die Bedingungen (1) und (3) bestimmt.  
\end{proof}

\begin{proposition}{(Supremum unter Streckung)}
	Sei $A \subseteq \mathbb{R}$ eine nicht-leere, nach oben beschränkte Teilmenge und sei $c > 0$. Dann ist $cA$ nach oben beschränkt und es gilt
	
	\[\sup(cA) = c \sup(A)\]
\end{proposition}

\begin{proof}
	Sei $s = \sup(A)$. Dann gitl $\forall a \in A, c > 0 : a \leq s \implies ca \leq cs$. Da aber jedes Element von $cA$ von der Form $ca$ für ein $a \in A$ ist. erhalten wir, dass $cs$ eine obere Schranke von $cA$ ist und dass $cA$ nach oben beschränkt ist.
	
	Sei $\varepsilon > 0$. Dann $\exists a \in A: a > s - \frac{e}{c}$, für welches die Ungleichung $ca > cs -\varepsilon$ gilt. Dies zeigt die zweite charakterisierende Eigenschaft des Supremums und wir erhalten $\sup (cA) = cs=c \sup (A)$.
\end{proof}

\begin{proposition}{(Supremum unter Summen)}
	Seien $A,B\subseteq \mathbb {R}$ zwei nicht-leere, von oben beschränkte Teilmengen von $\mathbb {R}$. Dann ist $A+B$ von oben beschränkt und es gilt 
	\[\sup(A+B) = \sup(A) + \sup(B)\]
\end{proposition}

\begin{proof}
	Wir definiere $s_A = \sup(A)$ und $s_B = \sup(B)$. Dann gilt $\forall a \in A: a \leq s_A$ und $\forall b in B: b \leq s_B$, was $\forall a \in A, b \in B: a + b \leq s_A + s_B$ impliziert. Da aber jedes Element von $A + B$ dieser Form ist, erhalten wir, dass $s_A + s_B$ eine obere schranke von $A + B$ ist und dass $A + B$ nach oben beschränkt ist.
	
	Sei $\varepsilon >0$. Dann existiert ein $a\in A$ mit $a > s_A - \frac {\varepsilon }{2}$ und ein $b\in B$ mit $b > s_B - \frac {\varepsilon }{2}$, was wiederum $a+b > s_A+s_B-\varepsilon$ impliziert. Dies zeigt die zweite charakterisierende Eigenschaft von $\sup (A+B)$ und wir erhalten 
	\[\sup(A + B) = s_A + s_B = \sup(A) + \sup(B)\]
\end{proof}


\subsection{Uneigentliche Werte, Suprema und Infime}

\subsection{Verwendung des Supremums und Infimums}
\end{document}