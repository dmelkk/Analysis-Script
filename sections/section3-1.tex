\documentclass[../Analysis1_script.tex]{subfiles}

\begin{document}

\begin{definition}{(Folge)}
	Sei $X$ eine Menge. Eine \textbf{Folge} in $X$ ist eine Abbildung $a: \mathbb {N} \to X$. Das Bild $a(n)$ von $n\in \mathbb {N}$ schreibt man auch als $a_n$ und bezeichnet es als das $n$-te \textbf{Folgenglied} von $a$. Anstatt $a:\mathbb {N}\to X$ schreibt man oft $(a_n)_{n\in \mathbb {N}}$, $(a_n)_{n=1}^\infty$ oder kurz $(a_n)_{n}$. Die Menge der Folgen in $X$ wird auch als $X^{\mathbb {N}}$ bezeichnet. Eine Folge $(a_n)_n$ heisst \textbf{konstant}, falls $a_n = a_m$ für alle $m,n\in \mathbb {N}$, und \textbf{schliesslich konstant}, falls ein $N\in \mathbb {N}$ existiert mit $a_n = a_m$ für alle $m,n\in \mathbb {N}$ mit $m,n \geq N$. 
\end{definition}

\begin{definition}{(Grenzwert)}
	Wir sagen, dass eine Folge $(a_n)_{n}$ in $\mathbb {C}$ gegen eine Zahl $A\in \mathbb {C}$ \textbf{strebt}, gegen die Zahl $A$ \textbf{konvergiert} oder \textbf{Grenzwert} $A$ hat, falls es für jedes $\varepsilon >0$ ein $N \in \mathbb {N}$ gibt, so dass $|a_n-A|< \varepsilon$ für alle $n \geq N$. In diesem Fall nennen wir die Folge \textbf{konvergent}. Falls es keinen Grenzwert in $\mathbb {C}$ gibt, nennen wir die Folge \textbf{divergent}. 
\end{definition}

In Prädikatenlogik ist Konvergenz gegen $A$ durch
	\[\begin{aligned}[]
		\forall \varepsilon >0\ \exists N \in \mathbb {N}\ \forall n \geq N: |a_n - A| < \varepsilon
	\end{aligned}\]
gegeben. Wir bemerken noch, dass eine Folge $(a_n)_n in \mathbb {C}$ genau dann gegen $A \in \mathbb {C}$ konvergiert, wenn die Folge $(|a_n-A|)_n$ gegen Null konvergiert.

\begin{lemma}{(Eindeutigkeit des Grenzwerts)}
	Der Grenzwert einer konvergenten Folge $(a_n)_{n}$ ist eindeutig bestimmt. Wir bezeichnen ihn mit $\lim _{n \to \infty }a_n$.
\end{lemma}

\begin{proof}
	Die Folge $(a_n)_{n}$ konvergiere gegen $A_1\in \mathbb {C}$ und $A_2\in \mathbb {C}$. Wir nehmen an, dass $A_1 \neq A_2$ und sei $\varepsilon = |A_1-A_2|> 0$. Dann können wir $N_1,N_2 \in \mathbb {N}$ finden, so dass für alle $n \geq N_1$ gilt $|a_n - A_1|< \frac {\varepsilon }{2}$ und für alle $n \geq N_2$ gilt $|a_n - A_2|< \frac {\varepsilon }{2}$. Für $N = \max \left \lbrace {N_1,N_2} \right \rbrace$ gelten also beide Aussagen und es folgt
	\[\begin{aligned}[]
		\varepsilon = |A_1-A_2| = |A_1-a_N+a_N-A_2| \leq |a_N-A_1|+|a_N-A_2|<\varepsilon .
	\end{aligned}\]
	Dies ist ein Widerspruch. Das heisst, $A_1 = A_2$ und das Lemma folgt. 
\end{proof}

\begin{lemma}{(Beschränkheit)}
	Jede konvergente Folge ist beschränkt. 
\end{lemma}

\begin{proof}
	Sei $(a_n)_{n}$ eine konvergente Folge und $A = \lim _{n\to \infty }a_n$. Dann existiert ein $N \in \mathbb {N}$, so dass $|a_n-A|<1$ für alle $n \geq N$. Daraus folgt
	\[\begin{aligned}[]
		|a_n| = |a_n - A+A| \leq |a_n-A|+|A| <1+|A|
	\end{aligned}\]
	für alle $n \geq N$ und
	\[\begin{aligned}[]
		|a_n| \leq \max \left \lbrace {|a_1|,|a_2|,\ldots ,|a_{N-1}|,1+|A|} \right \rbrace
	\end{aligned}\]
	für alle $n \in \mathbb {N}$.
\end{proof}

\begin{proposition}{(Additive und Multiplikative Eigenschaften des Grenzwertes)}\label{folg:add_and_mult}
	Seien $(a_n)_n, (b_n)_n$ zwei konvergente Folgen.
	\begin{enumerate}
		\item Die Folge $(a_n)_n + (b_n)_n$ ist konvergent und es gilt
			\[\lim_{n \to \infty}(a_n + b_n) = \lim_{n \to \infty} a_n + \lim_{n \to \infty} b_n . \]
		\item Die Folge $(a_n b_n)_n$ ist konvergent und es gilt
			\[\lim_{n \to \infty}(a_n b_n) = (\lim_{n \to \infty} a_n)(\lim_{n \to \infty} b_n) . \]
			Insbesondere ist für $\alpha \in \mathbb{R}$ die Folge $\alpha (a_n)_n$ konvergent und 
			\[\lim_{n \to \infty}(\alpha a_n) = \alpha \lim_{n \to \infty} a_n . \]
		\item Angenommen $a_n \neq 0$ für alle $n \in \mathbb{N}$ und $\lim_{n \to \infty} a_n \neq 0$. Dann ist die Folge $(\frac{1}{a_n})_n$ konvergent und es gilt
			\[\lim_{n \to \infty} \frac{1}{a_n} = \frac{1}{\lim_{n \to \infty} a_n}\]
	\end{enumerate}
	Nach (i) und (ii) bildet die Menge der konvergenten Folgen in $\mathbb{C}^\mathbb{N}$ einen Unterraum und der Grenzwert stellt eine lineare Abbildung von diesem Unterraum nach $\mathbb{C}$ dar.
\end{proposition}
 
\begin{proof}
	Für (i) und (ii) angenommen wi $a_n \neq 0$ für alle $n \in \mathbb{N}$ und $A = \lim_{n \to \infty} a_n \neq 0$. Dann gilt
	\[\begin{aligned}[]
		\bigg |\frac {1}{a_n}- \frac {1}{A} \bigg | = \frac {|A-a_n|}{|a_nA|}
	\end{aligned}\]
	Wir sehen also, dass wir erzwingen können, dass $\big |\frac {1}{a_n}- \frac {1}{A} \big |$ klein ist, wenn $|A-a_n|$ klein ist. Dazu müssen wir allerdings verhindern, dass $a_n$ zu klein wird. Für den formalen Beweis sei $\varepsilon >0$. Nach Definition von $A=\lim _{n \to \infty }a_n$ existiert ein $N \in \mathbb {N}$, so dass
	\[\begin{aligned}[]
		|a_n-A| < \min \left \lbrace {\frac {|A|}{2},\frac {\varepsilon |A|^2}{2}} \right \rbrace
	\end{aligned}\]
	für alle $n \geq N$. Für $n \geq N$ gilt dann nach der umgekehrten Dreiecksungleichung
	\[\begin{aligned}[]
		|a_n| = |a_n -A + A| \geq |A| - |a_n-A| > |A| - \frac {|A|}{2} = \frac {|A|}{2}.
	\end{aligned}\]
	Also wird $a_n$ nicht zu klein und
	\[\begin{aligned}[]
		\bigg |\frac {1}{a_n}- \frac {1}{A} \bigg | = \frac {|A-a_n|}{|a_n||A|} < \frac {|a_n-A|}{|A|^2/2} < \frac {\varepsilon |A|^2/2}{|A|^2/2} = \varepsilon ,
	\end{aligned}\]
	was zu zeigen war.
\end{proof}

Eine konvergente Folge mit Grenzwert Null wird auch eine \textbf{Nullfolge} genannt.

\begin{lemma}{(Indexverschiebung)}
	Für eine konvergente Folge $(a_n)_n$ und $\ell \in \mathbb{N}_0$ ist die Folge $(a_{n + \ell})_n$ kovergent und es gilt
	\[\lim_{n \to \infty} a_n = \lim_{n \to \infty} a_{n + \ell}\]
\end{lemma}

\subsection{Zusammenhang zur Stetigkeit}

Stetigkeit lässt sich auch mit Hilfe von Folgen charakterisieren.

\begin{proposition}{(Folgenstetigkeit bei einem Punkt)}
	Sei $D \subseteq \mathbb{C}$ eine Teilmenge, $f : D \to \mathbb{C}$ eine Funktion und $z_0 \in D$. Die Funktion ist genau dann stetig bei $z_0$, wenn für jede Folge $(a_n)_n$ in $D$ mit $\lim_{n \to \infty}a_n z_0$ auch $\lim_{n \to \infty}f(a_n) = f(z_0)$ gilt.
\end{proposition}

\begin{proof}
	Angenommen $f$ ist bei $z_0$ stetig und $(a_n)_n$ ist eine Folge in $D$ mit $\lim _{n\to \infty }a_n = z_0$. Dann existiert für $\varepsilon >0$ ein $\delta >0$ mit
	\[\begin{aligned}[]
		|z-z_0|<\delta \implies |f(z)-f(z_0)| < \varepsilon .
	\end{aligned}\]
	Des Weiteren existiert ein $N \in \mathbb {N}$ mit
	\[\begin{aligned}[]
		n \geq N \implies |a_n - z_0| < \delta ,
	\end{aligned}\]
	was gemeinsam
	\[\begin{aligned}[]
		n \geq N \implies |f(a_n) - f(z_0)| <\varepsilon
	\end{aligned}\]
	ergibt. Die Folge $(f(a_n))_n$ konvergiert also gegen $f(z_0)$.
	
	Für die Umkehrung nehmen wir an, dass $f$ nicht stetig ist in $z_0$. Dann existiert ein $\varepsilon >0$, so dass für alle $\delta >0$ ein $z\in D$ existiert mit
	\[\begin{aligned}[]
		|z-z_0| < \delta \wedge |f(z)-f(z_0)| \geq \varepsilon
	\end{aligned}\]
	 Wir verwenden dies für $n\in \mathbb {N}$ und $\delta = \frac {1}{n}>0$ und finden also ein $a_n \in D$ mit
	\begin{align}\label{eq:folg-seqcontpf1} 
		|a_n -z_0|< \frac {1}{n}
	\end{align}
	und
	\begin{align}\label{eq:folg-seqcontpf2} 
		|f(a_n)-f(z_0)| \geq \varepsilon .
	\end{align}
	Aus Ungleichung (\ref{eq:folg-seqcontpf1}) schliessen wir, dass die Folge $(a_n)_n$ gegen $z_0$ konvergiert und aus Ungleichung (\ref{eq:folg-seqcontpf2}) folgt, dass $(f(a_n))_n$ nicht gegen $f(z_0)$ konvergiert.
\end{proof}

\subsection{Teilfolgen}

\begin{definition}{(Teilfolge)}
	Wenn $(a_n)_n$ eine Folge ist und $(n_k)_k:k \in \mathbb {N} \to n_k \in \mathbb {N}$ eine streng monoton wachsende Folge ist, dann wird $(a_{n_k})_k$ eine \textbf{Teilfolge} von $(a_n)_n$ genannt. 
\end{definition}

\begin{lemma}{(Konvergenz von Teilfolgen)}
	Sei $(a_n)_n$ eine konvergente Folge. Jede Teilfolge $(a_{n_k})_k$ von $(a_n)_n$ konvergiert und hat denselben Grenzwert $\lim _{k\to \infty }a_{n_k} = \lim _{n\to \infty } a_n$. 
\end{lemma}

Eine Folge kann konvergente Teilfolgen besitzen, ohne selbst zu konvergieren. Beispielsweise hat die Folge $n \in \mathbb {N} \mapsto \ (-1)^n \in \mathbb {R}$ die konvergente (konstante) Teilfolge $n \in \mathbb {N} \mapsto (-1)^{2n}\in \mathbb {R}$, konvergiert aber nicht, wie wir schon gesehen haben. 

\begin{proposition}{(Häufungspunkte einer Folge)}
	Sei $(a_n)_n$ eine Folge in $\mathbb {C}$. Eine Zahl $A \in \mathbb {C}$ heisst \textbf{Häufungspunkt} von $(a_n)_n$, falls die folgenden äquivalenten Bedingungen erfüllt sind.
	\begin{enumerate}
		\item Es gibt eine Teilfolge $a_{n_k}$, so dass $\lim _{k \to \infty }a_{n_k} = A$. 
		\item Für alle $\varepsilon >0$ und $N\in \mathbb {N}$ gibt es ein $n\geq N$ mit $|a_n-A|<\varepsilon$.
	\end{enumerate} 
\end{proposition}

\begin{proof}
	Angenommen (a) gilt. Sei also $(a_{n_k})_k$ eine konvergente Teilfolge von $(a_n)_n$ mit Grenzwert $A$ und sei $\varepsilon > 0$. Dann existiert ein $K\in \mathbb {N}$ mit $|a_{n_k}-A|<\varepsilon$ für alle $k \geq K$. Sei nun $k \geq K$ mit $n_k \geq N$. Dann erfüllt $n = n_k$ die Bedingung $|a_{n}-A|<\varepsilon$ wie gewollt und (b) ist erfüllt.
	
	Angenommen (b) gilt. Wir möchten rekursiv eine Teilfolge $(a_{n_k})_k$ finden mit
	\[\begin{aligned}[]
		|a_{n_k}-A| < \frac {1}{k}
	\end{aligned}\]
	für alle $k\in \mathbb {N}$. Diese konvergiert dann gegen $A$, da für $\varepsilon >0$ die Ungleichung $|a_{n_\ell }-A| < \varepsilon$ für alle $\ell > \frac {1}{\varepsilon }$ erfüllt ist.
	
	Sei $\varepsilon = 1$ und $N=1$. Dann gibt es ein $n_1 \geq N=1$ mit $|a_{n_1}-A|<1$. Nun nehmen wir an, dass $n_1<n_2 < \ldots <n_k$ bereits konstruiert sind mit
	\[\begin{aligned}[]
		|a_{n_\ell }-A| < \frac {1}{\ell }
	\end{aligned}\]
	für $\ell =1, \ldots , k$. Wir setzen $\varepsilon = \frac {1}{k+1}$ und $N = n_k+1$. Dann existiert nach Voraussetzung ein $n_{k+1} \geq N > n_k$ mit
	\[\begin{aligned}[]
		|a_{n_{k+1}}-A| < \frac {1}{k+1}.
	\end{aligned}\] 
	Dies beendet den Induktionsschritt und wir erhalten durch Rekursion die gewünschte Teilfolge $(a_{n_k})_k$ mit Grenzwert $A$. 
\end{proof}

\subsection{Cauchy-Folgen}

\begin{definition}{(Cauchy-Folge)}
	Eine Folge $(a_n)_n$ in $\mathbb{C}$ ist eine Cauchy-Folge, fallse es für jedes $\varepsilon > 0$ ein $N \in \mathbb{N}$ gibt, so dass
	\[|a_m - a_n| < \varepsilon\]
	für alle $m, n \geq N$.
\end{definition}

\subsection{Reduktion auf reelle Folgen}

Anstelle von Folgen in $\mathbb{C}$ ist es meistens ausreichend, Folgen in $\mathbb{R}$ zu betrachten.

\begin{lemma}{(Reduktion})
	Eine komplexwertige Folge $(a_n)_n$ ist genau dann konvergent (mit Grenzwert $a\in \mathbb {C}$), wenn die beiden reellwertigen Folgen $(\operatorname {Re}(a_n))_n$ und $(\operatorname {Im}(a_n))_n$ konvergent sind (mit Grenzwerten $\operatorname {Re}(a)$ respektive $\operatorname {Im}(a)$). 
\end{lemma}

\begin{proof}
	Angenommen $(a_n)_n$ konvergiert gegen $a\in \mathbb {C}$. Für $\varepsilon >0$ existiert dann ein $N \in \mathbb {N}$, so dass $|a_n-a|< \varepsilon$ für $n\in \mathbb {N}$. Da für alle $b\in \mathbb {C}$ gilt $|\operatorname {Re}(b)| \leq |b|$ und $|\operatorname {Im}(b)| \leq |b|$, folgt
	\[\begin{aligned}[]
		|\operatorname {Re}(a_n)-\operatorname {Re}(a)| &= |\operatorname {Re}(a_n-a)| \leq |a_n -a| < \varepsilon \\ |\operatorname {Im}(a_n)-\operatorname {Im}(a)| &= |\operatorname {Im}(a_n-a)| \leq |a_n -a| < \varepsilon
	\end{aligned}\]
	für alle $n \geq N$. Dies zeigt wie gewünscht, dass die Folgen $(\operatorname {Re}(a_n))_n$ und $((\operatorname {Im}(a_n))_n$ gegen $\operatorname {Re}(a)$ respektive $\operatorname {Im}(a)$ konvergieren.
	
	Die Umkehrung folgt aus $a_n = \operatorname {Re}(a_n) + \operatorname {Im}(a_n)\mathrm {i}$ für alle $n\in \mathbb {N}$ und Proposition \ref{folg:add_and_mult}(i) (Additive Eigenschaft des Grenzwerts) 	
\end{proof}
\end{document}