\documentclass[../Analysis1_script.tex]{subfiles}
\begin{document}

Sei $n \in \mathbb {N}$ und seien $a_1,\ldots ,a_n \in \mathbb {C}$ oder $a_1,\ldots ,a_n$ Elemente eines Vektorraums $V$ (wie zum Beispiel $\mathbb {R}^d$ für ein $d \geq 1$). Wir wollen hier für eine natürliche Zahl $n\in \mathbb {N}$ die Summe von $a_1$ bis $a_n$, also
\[\begin{aligned}[]
	\sum _{j=1}^n a_j = a_1+\ldots +a_n,
\end{aligned}\]
besprechen und formal korrekt definieren.

Vom formalen Standpunkt her gesehen ist $j \mapsto a_j\in V$ eine Funktion, (die oft auch durch eine konkrete Formel gegeben sein wird und) deren Definitionsbereich die Menge
\[\begin{aligned}[]
	\left \lbrace {j \in \mathbb {N}} \mid {1 \leq j \leq n}\right \rbrace
\end{aligned}\]
enthalten muss. Wir können $\sum _{i=1}^n a_j$ rekursiv definieren durch
\[\begin{aligned}[]
	\sum _{j=1}^1 a_j = a_1\text { und } \sum _{j=1}^{k+1} a_j = \bigg (\sum _{j=1}^{k} a_j\bigg ) + a_{k+1}
\end{aligned}\]
für $k \in \left \lbrace {1,\ldots ,n-1} \right \rbrace$. Diese Definition entspricht einem einfachen rekursiven Algorithmus, um die Summe $\sum _{i=1}^n a_j$ zu berechnen. Allgemeiner ist die Summe $\sum _{i=m}^n a_j$ für ganze Zahlen $m,n$ ebenso rekursiv durch
\[\begin{aligned}[]
	\sum _{j=m}^na_j=\begin{cases}
		0 &\text {falls }m>n,\\ 
		a_m &\text {falls }m=n\text { und}\\ 
		\big (\sum _{j=m}^{n-1} a_j\big ) + a_{n}&\text {falls }m<n
	\end{cases}
\end{aligned}\]
definiert. Wir werden $a_j$ als die \textbf{Summanden} und $j$ als den \textbf{Index} der Summe $\sum _{j=1}^n a_j$ bezeichnen.  

Falls nun $m,n$ ganze Zahlen und $a_m,\ldots ,a_n \in \mathbb {C}$ sind, dann können wir auch das \textbf{Produkt} $\prod _{j=m}^n a_j$ von $a_m$ bis $a_n$ rekursiv durch
\[\begin{aligned}[]
	\prod _{j=m}^na_j=\begin{cases}
		1 &\text {falls }m>n,\\ 
		a_m &\text {falls }m=n\text { und}\\ 
		\big (\prod _{j=m}^{n-1} a_j\big ) \cdot a_{n}&\text {falls }m<n
	\end{cases}
\end{aligned}\]
definieren. Wir werden $a_j$ als die \textbf{Faktoren} und $j$ als den \textbf{Index} des Produkts $\prod _{j=m}^n a_j$ bezeichnen. 


\subsection{Rechenregeln für die Summe}

Die Summe erfüllt für gegebene ganze Zahlen $m,n$ mit $m \leq n$ die Gleichungen
\[\begin{aligned}[]
	\sum _{k=m}^n(a_k+b_k) = \sum _{k=m}^na_k + \sum _{k=m}^nb_k
\end{aligned}\]
und
\[\begin{aligned}[]
	\sum _{k=m}^n(ca_k) = c \sum _{k=m}^na_k,
\end{aligned}\] 
wobei $a_m,\ldots ,a_n, b_m,\ldots ,b_n$ in einem reellen (respektive komplexen) Vektorraum $V$ liegen und $c\in \mathbb {R}$ (respektive $c\in \mathbb {C}$) ein Skalar ist. (Die erste Eigenschaft ist eine Mischung aus Assoziativgesetz und Kommutativgesetz für die Addition, und die zweite Eigenschaft ist eine Verallgemeinerung des Distributivgesetzes.)

Diese beiden Eigenschaften (die Summe wird auf die Summe und das skalare Vielfache auf das skalare Vielfache abgebildet) werden auch als \textbf{Linearität der Abbildung} $\sum$ bezeichnet, wobei $\sum$ auf dem Vektorraum $V^{\{ m,\ldots ,n\} }$ der Funktionen von $\{ m,\ldots ,n\}$ nach $V$ definiert ist, den Vektorraum V als Zielbereich besitzt, und $(a_m,\ldots ,a_n)\in V^{\{ m,\ldots ,n\} }$ auf $\sum _{k=m}^na_k$ abbildet. Wie der Name sagt, wird Linearität ausführlicher in der Linearen Algebra besprochen. Es handelt sich dabei aber auch um eine wichtige Eigenschaft für die Analysis, welche also häufig auftreten.

Des Weiteren gilt die Formel für die \textbf{Teleskopsumme}
\[\begin{aligned}[]
	\sum _{k=m}^n(a_{k+1}-a_k) &= (a_{m+1}-a_m) + (a_{m+2}-a_{m+1})+\ldots + (a_n-a_{n-1})+(a_{n+1}-a_n)\\ 
	&= a_{n+1}-a_m
\end{aligned}\] 
wobei $a_m,\ldots ,a_{n+1}$ in einem reellen oder einem komplexen Vektorraum liegen. Formaler argumentiert gilt
\[\begin{aligned}[]
	\sum _{k=m}^n(a_{k+1}-a_k) &= \sum _{k=m}^na_{k+1}-\sum _{k=m}^na_k = \sum _{j=m+1}^{n+1}a_j - \sum _{k=m}^n a_k\\ 
	&= \Bigg (a_{n+1}+ \sum _{j=m+1}^na_j\Bigg ) - \Bigg (a_m + \sum _{k=m+1}^na_k\Bigg ) = a_{n+1}-a_m
\end{aligned}\] 
wie bereits behauptet. Die Formel für die Teleskopsumme lässt sich zur Abel-Summationsformel verallgemeinern, welche überraschend viele Anwendungen in der Analysis und Zahlentheorie findet.


\subsection{Rechenregeln für die Produkt}

Für ganze Zahlen $m \leq n$ und $a_m,\ldots ,a_n,b_m,\ldots ,b_n \in \mathbb {C}$ gilt
\[\begin{aligned}[]
	\prod _{k=m}^n (a_kb_k) = \bigg (\prod _{k=m}^n a_k\bigg ) \bigg (\prod _{k=m}^n b_k\bigg ).
\end{aligned}\]
Insbesondere ist für alle $c\in \mathbb {C}$
\[\begin{aligned}[]
	\prod _{k=m}^n (ca_k) = c^{n-m+1}\bigg (\prod _{k=m}^n a_k\bigg ).
\end{aligned}\] 
Des Weiteren gilt für alle $a_m,\ldots ,a_n$ die Formel für das \textbf{Teleskopprodukt}
\[\begin{aligned}[]
	\prod _{k=m}^n \frac {a_{k+1}}{a_k} &= \bigg (\prod _{k=m}^n a_{k+1}\bigg )\bigg ( \prod _{k=m}^n \frac {1}{a_k}\bigg ) = \bigg (\prod _{k=m+1}^{n+1} a_{k}\bigg )\bigg ( \prod _{k=m}^n \frac {1}{a_k}\bigg )\\ 
	&= a_{n+1}\bigg (\prod _{k=m+1}^{n} a_{k}\bigg )\bigg ( \prod _{k=m+1}^n \frac {1}{a_k}\bigg )\frac {1}{a_m} = \frac {a_{n+1}}{a_m}.
\end{aligned}\]

\begin{lemma}{(Bernoulli'sche Ungleichnug)}
	Für alle reellen Zahlen $a \geq -1$ und $n \in \mathbb{N}_0$ gilt $(1 + a)^n \geq 1 + na$.
\end{lemma}

\begin{proof}
	Wir verwänden vollständige Induktion. Für $n = 0$ haben wir $(1 + a)^n = 1 = 1 + na$. Angenommen die Ungleichung $(1 + a) ^ n \geq 1 + na$ gilt für ein $n \in \mathbb{N}_0$. Nach Annahme an $a$ ist $a \geq -1$, was in Kombination mit Annahme an $n$ 
	\[\begin{aligned}[]
		(1 + a)^n &= (1 + a)^n (1 + a) \\
			&\geq (1 + na)(1 + a) = 1 + na + a + na^2 \\
			&\geq 1 + (n+1)a
	\end{aligned}\]
	ergibt und damit Induktionsschritt zeigt. Das Lemma folgt. 
\end{proof}

\subsection{Die geometrische Summe}
\begin{proposition}{(Geometrische Summenformel)}
	Sei $n \in \mathbb {N}_0$ und $q \in \mathbb {C}$. Dann gilt
	\[\begin{aligned}[]
		\sum _{k=0}^n q^k = \left \lbrace 
		\begin{array}{cc} 
			n+1 & \text {falls } q=1 \\ 
			\frac {q^{n+1}-1}{q-1} & \text {falls } q \neq 1
		\end{array} \right . .
	\end{aligned}\] 
\end{proposition}
Der direkte (aber sicher nicht eleganteste) Beweis verwendet vollständige Induktion:
\begin{proof}
	Für $q=1$ ist $q^k=1$ für alle $k\in \mathbb {N}_0$ und die Aussage folgt aus den Eigenschaften der Summe. Sei nun $q\neq 1$. Für $n=0$ gilt $\sum _{k=0}^0 q^k = q^0 = 1 = \frac {q-1}{q-1}$, was also den Induktionsanfang zeigt. Angenommen die Formel in der Proposition gilt bereits für $n$. Dann ist
	\[\begin{aligned}[]
		\sum _{k=0}^{n+1} q^k = \sum _{k=0}^n q^k + q^{n+1} = \frac {q^{n+1}-1}{q-1} + q^{n+1} = \frac {q^{n+1}-1}{q-1} + \frac {q^{n+2}-q^{n+1}}{q-1} = \frac {q^{n+2}-1}{q-1},
	\end{aligned}\]
	womit der Induktionsschritt gezeigt ist und die Proposition folgt.   
\end{proof}
\end{document}