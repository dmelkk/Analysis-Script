\documentclass[../Analysis1_script.tex]{subfiles}
\begin{document}


\subsection{Intervalle}

\begin{definition}{(Intervalle)}
	Seien $a, b \in \mathbb{R}$. Dann ist die \textbf{abgeschlossene Intervall} $[a, b]$ durch
	\[ [a,b] = \{x \in \mathbb{R} | a \leq x \leq b\} \]
	das \textbf{offene Intervall} $(a, b)$ durch
	\[ (a,b) = \{x \in \mathbb{R} | a < x < b\} \]
	das \textbf{(rechts) halboffene Intervall} $[a, b)$ durch
	\[ [a,b) = \{x \in \mathbb{R} | a \leq x < b\} \]
	und das \textbf{(links) halboffene Intervall} $(a, b]$ durch
	\[ (a,b] = \{x \in \mathbb{R} | a < x \leq b\} \]
	definiert. Wenn das Intervall nicht-leer ist, dann wird $a$ der \textbf{linke Endpunkt}, $b$ der \textbf{rechte Endpunkt}, und $b - a$ die \textbf{Länge des Intervalls} gennant.
	
	Die Intervalle $(a, b), (a, b], [a, b)$ für $a, b \in \mathbb{R}$ nicht-leer sind genau dann, wenn $a < b$, und $[a, b]$ nicht-leer ist genau dann, wenn $a \leq b$. Intervalle der Art $(a, b), (a, b], [a, b), [a, b]$ für $a, b \in \mathbb{R}$ werden auch \textbf{endliche} oder \textbf{beschränkte Intervalle} genant.
\end{definition}

\begin{definition}{(Unbeschränkte Intervalle)}
	Für $a, b \in \mathbb{R}$ definieren wir die \textbf{unbeschränkte abgeschlossenen Intervalle}
	\begin{align*}
		[a, \infty) &= \mathbb{R}_{\geq a} = {x \in \mathbb{R} | x \geq a}\\
		(-\infty, b] &= \mathbb{R}_{\leq b} = {x \in \mathbb{R} | x \leq b}
	\end{align*}
	und \textbf{undeschränkte offenen Intervalle}
	\begin{align*}
		(a, \infty) &= \mathbb{R}_{> a} = {x \in \mathbb{R} | x > a}\\
		(-\infty, b) &= \mathbb{R}_{< b} = {x \in \mathbb{R} | x < b}\\
		(-\infty, \infty) &= \mathbb{R}
	\end{align*}
\end{definition}

\begin{definition}{(Umgebung eines Punktes)}
	Sei $x \in \mathbb{R}$. Eine Menge, die ein offenes Intervall enthält, in dem $x$ liegt, wird auch \textbf{Umgebung} von $x$ gennan. Für ein $\delta > 0$ wird das offene Intervall $(x - \delta, x + \delta)$ die $\delta$-\textbf{Umgebung} genannt.
\end{definition}

\subsection{Der Absolutbetrag auf den reellen Zahlen}

\begin{definition}{(Der Absolutbetrag)}
	Der \textbf{Absolutbetrag} ist die Funktion
	\[|\cdot| : \mathbb{R} \to \mathbb{R}, x \to |x| = 
		\begin{cases}
			x, &\text{falls } x \geq 0\\
			-x, &\text{falls } x < 0
		\end{cases}
	\]
	Wir betrachten zuerst einige Konsequenzen dieser Definition.
\end{definition}

\begin{corollary}
	Für $x \in \mathbb{R}$ ist $|x| \geq 0$ und $|x| = 0$ genau dann, wenn $x = 0$. Dies folgt aus der Trichotomie von reellen Zahlen: Für $x = 0$ gilt $|x| = 0$, für $x > 0$ gilt $|x| = x > 0$ und für $x < 0$ folgt $|x| = -x > 0$.
\end{corollary}

\begin{corollary}
	Es ist $|x| = |-x|$ für alle $x \in \mathbb{R}$
\end{corollary}

\begin{corollary}\label{cor:mult}
	Die Absolutbetrag ist multiplikativ: $\forall x, y \in \mathbb{R}: |xy| = |x||y|$
\end{corollary}

\begin{corollary}
	$\forall x \in \mathbb{R}^{\times} = \mathbb{R} \setminus \{0\}: |\frac{1}{x}| = \frac{1}{|x|}$. Deis folgt aus \ref{cor:mult} wegen $\forall x \in \mathbb{R} \setminus\{0\}: |\frac{1}{x}||x| = 1$ 
\end{corollary}

\begin{corollary}
	$\forall x, y \in \mathbb{R}: |x| \leq y \iff -y \leq x \leq y$. Denn angenommen $|x| \leq y$. Falls $x \geq 0$ dann gilt $-y \leq 0 \leq x = |x| \leq y$. Falls $x < 0$, dann ist $-y \leq -|x| = x < 0 \leq y$ und damit wiederum $-y \leq x \leq y$. Für die Umkehrung bemerken wir, dass $-y \leq x \leq y$ auch $-y \leq -x \leq y$ und somit in jedem Fall $|x| \leq y$ impliziert.
\end{corollary}

\begin{corollary}
	Ananlog ist $\forall x, y \in \mathbb{R}: |x| < y \iff -y < x < y$
\end{corollary}

\begin{corollary}{(Dreiecksungleichung)}
	\[\forall x, y \in \mathbb{R}: |x + y| \leq |x| + |y|\]
	Diese Ungleichung wird auch die \textbf{Dreiecksungleichung} gennant. Sie folgt, in dem wir $-|x| \leq x \leq |x|$ und $-|y| \leq y \leq |y|$ und anschliessend auf
	\[-(|x| + |y|) \leq x + y \leq |x| + |y|\]
	anwenden.
\end{corollary}

\begin{corollary}{(umgekehrte Dreiecksungleichung)}
	\[\forall x, y \in \mathbb{R}: \big||x| - |y|\big| \leq |x - y|\]
	Denn Dreiecksungleichung zeigt:
	\[|x| \leq |x - y + y| \leq |x - y| + |y|\]
	was zu $|x| - |y| \leq |x - y|$ führt. Durch Vertauschen von $x, y$ erhalten wir $|y| - |x| \leq |x - y|$. Also $\big||x| - |y|\big| \leq |x - y|$ wie gewunscht. 
\end{corollary}

\begin{definition}{(Offene und abgeschlossene Teilmenge)}
	Ein Teilmanege $U \subseteq \mathbb{R}$ heisst \textbf{offen} (in $\mathbb{R}$), wenn 
	\[\forall x \in U \exists \varepsilon > 0 : (x - \varepsilon, x + \varepsilon) \subseteq U\]
	
	Ein Teilmaenge heisst \textbf{abgeschlossen} (in $\mathbb{R}$), wenn ihr Komplement $\mathbb{R} \setminus A$ offen ist.
\end{definition}


\subsection{Der Absolutbetrag auf den komplexen Zahlen}

\begin{definition}{(Der Absolutbetrag auf $\mathbb{C}$)}
	Der \textbf{Absolutbetrag $|\cdot|$ auf $\mathbb{C}$} ist gegeben durch
	\[|x + y\mathrm{i}| = \sqrt{x^2 + y^2}\]
	für $x + y\mathrm{i} \in \mathbb{C}$
	
	An dieser Stelle bemerken wir, dass für $z = x + y\mathrm{i} \in \mathbb{C}$ die Summe der Quadrate $x^2 + y^2$ gerade gleich $z\bar{z}$ ist, denn
	\[(x + y\mathrm{i})(x - y\mathrm{i}) = x^2 + y^2 + (xy -xy)\mathrm{i} = x^2 + y^2\]
	Somit gilt für alle $z \in \mathbb{C}$ 
	\[|z| = z\bar{z}\]
\end{definition}

\subsubsection{Eigenschaften des Absolutbetrags auf $\mathbb{C}$}
	\begin{enumerate}
		\item{(Definitheit)} $\forall z \in \mathbb{C}: |z| \geq 0 \wedge |z| = 0 \iff z = 0$
		\item{(Multiplikativität)} $\forall z, w \in \mathbb{C} : |zw| = |z||w|$
		\item{(Dreiecksgleichung)} $\forall z, w \in \mathbb{C} : |z + w| \leq |z| + |w|$
		\item{(Umgekehrte Dreiecksgleichung)} $\forall z, w \in \mathbb{C} : \big||z| - |w|\big| \leq |z - w|$
	\end{enumerate}
	
	\textbf{*** Hier muss Beweis sein ****}

\begin{definition}{(Offene Bälle)}
	Der \textbf{offene Ball} mit Radius $r > 0$ um einen Punkt $z \in \mathbb{C}$ ist die Menge
	\[B_r(z) = \{w \in \mathbb{C}\big| |z - w| < r\}\]
	Der offene Ball $B_r(z)$ zu $r > 0$ und $z \in \mathbb{C}$ besteht also aus jeden Punkten, die Abstand (strikt) kleiner als $r$ von $z$ haben. Offene Bälle in $\mathbb{C}$ und offene Intervalle in $\mathbb{R}$ sind in folgendem Sinne kompatibelt: Ist $x \in \mathbb{R}$ und $r > 0$, so ist der Schnitt des offenes Balles $B_r(z) \in \mathbb{C}$ mit $\mathbb{R}$ gerade das offene, symmetrisch um $x$ liegende Itervall $(x - r, x + r)$. 
	
	\includesvg{../images/balls}
\end{definition}

	

\begin{definition}{(Offene und abgeschlossene Teilmanege von $\mathbb{C}$)}
	Eine Teilmenge $U \subseteq \mathbb{C}$ heisst $offen$ (in $\mathbb{C}$), wenn zu jedem Punkt in $U$ ein offene Ball um diesen Punkt exestiert, der in $U$ enthalten ist. Formaler:
	\[\forall z \in U \exists r > 0 : B_r(z) \subseteq U\]
	Eine Teilmenge $A \subseteq \mathbb{C}$ heisst \textbf{abgeschlossen} (in $\mathbb{C}$), falls ihr Komplement $\mathbb{C} \setminus A$ offen ist.
\end{definition}

\end{document}