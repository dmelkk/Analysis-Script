\documentclass[../Analysis1_script.tex]{subfiles}
\begin{document}

\subsection{Beschränktheit}

\begin{definition}{(Beschränktheit von Funktionen)}
	Sei $D$ eine nicht-leere Menge und sei $f: D \to \mathbb {R}$ eine Funktion. Wir sagen, dass die Funktion $f$
	\begin{itemize}
		\item{\textbf{von oben beschränkt}} ist, falls die Wertemenge $f(D)$ von oben beschränkt ist, 
		\item{\textbf{von unten beschränkt}} ist, falls die Wertemenge $f(D)$ von unten beschränkt ist, 
		\item{\textbf{beschränkt}} ist, falls $f$ von oben und von unten beschränkt ist.
	\end{itemize}
\end{definition}

\subsection{Monotonie}

\begin{definition}{(Monotonieeigenschaften)}
	Eine Funktion $f: D \to \mathbb{R}$ ist
	\begin{itemize}
		\item{\textbf{monoton wachsend}}, falls $\forall x, y \in D: x \leq y \implies f(x) \leq f(y)$,
		\item{\textbf{streng monoton wachsend}}, falls $\forall x, y \in D: x < y \implies f(x) < f(y)$,
		\item{\textbf{monoton fallsend}}, falls $\forall x, y \in D: x \leq y \implies f(x) \geq f(y)$,
		\item{\textbf{strend monoton fallend}}, falls $\forall x, y \in D: x < y \implies f(x) > f(y)$,
		\item{\textbf{monoton}}, falls $f$ monoton wachsend oder monoton fallend ist,
		\item{\textbf{streng monoton}}, falls $f$ streng monoton wachsend oder streng monoton fallend ist.
	\end{itemize}
\end{definition}
\end{document}